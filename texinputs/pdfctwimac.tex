% standard macros for CWEB listings (in addition to plain.tex)
% Version 4.9 --- June 2022
% modified for pages produced by CTWILL
% further modified for page size of the MMIXware book
% modified for PDF output with 'tex "\let\pdf+', 'pdftex', and 'xetex'
% This file is not copyrighted and can be used freely.
\ifx\renewenvironment\undefined\else\endinput\fi % LaTeX will use other macros
\xdef\fmtversion{\fmtversion+CTWILL4.9+LNCS+PDF}
\input iftex.sty % TeX engine tests
\ifx\pdf+\pdftrue\fi % for plain TeX in combination with dvipdfm
% Uncomment the following line if you want PDF goodies to be the default
%\ifx\pdf-\else\pdftrue\fi
\def\pdflinkcolor{0 0 1} % the RGB values for hyperlink color
\ifxetex\pdftrue\fi % XeTeX produces PDF output
\ifpdftex % pdfTeX produces PDF output if \pdfoutput>0
  \def\Black{\pdfliteral{0 g 0 G}}  % use rgb colors for direct PDF output too
  \def\PDFlinkcolor{\pdfliteral{\pdflinkcolor\space rg \pdflinkcolor\space RG}}
\fi
\let\ifacro=\ifpdf

\let\:=\. % preserve a way to get the dot accent
 % (all other accents will still work as usual)

\newdimen\em \em=10pt % this "em" will not change with font size
\parskip 0pt plus .1pt % almost no stretch between paragraphs
\parindent 1\em % for paragraphs and for the first line of C text

\newif\ifsorted
\newread\sreffile
\newwrite\reffile
\ifx\norefs\def\else
  \openin\sreffile=\jobname.sref
  \ifeof\sreffile \immediate\openout\reffile=\jobname.ref
       \sortedfalse \message{This is the first pass!}
  \else \sortedtrue \message{This is the second pass!} \fi
\fi

\font\ninerm=cmr9
\let\mc=\ninerm % medium caps
\def\CEE/{{\mc C\spacefactor1000}}
\def\UNIX/{{\mc U\kern-.05emNIX\spacefactor1000}}
\def\TEX/{\TeX}
\def\CPLUSPLUS/{{\mc C\PP\spacefactor1000}}
\def\Cee{\CEE/} % for backward compatibility
\def\Cpp{\CPLUSPLUS/} % for backward compatibility
\def\9#1{}
\font\eightrm=cmr8
\font\sixrm=cmr6
\font\ninei=cmmi9
\font\eighti=cmmi8
\font\sixi=cmmi6
\skewchar\ninei='177 \skewchar\eighti='177 \skewchar\sixi='177
\font\ninesy=cmsy9
\font\eightsy=cmsy8
\font\sixsy=cmsy6
\skewchar\ninesy='60 \skewchar\eightsy='60 \skewchar\sixsy='60
\font\ninebf=cmbx9
\font\eightbf=cmbx8
\font\sixbf=cmbx6
\font\ninett=cmtt9
\font\eighttt=cmtt8
\hyphenchar\ninett=-1 \hyphenchar\eighttt=-1
\font\ninesl=cmsl9
\font\eightsl=cmsl8
\font\nineit=cmti9
\font\eightit=cmti8
\font\tentex=cmtex10
\font\ninetex=cmtex9 % TeX extended character set (used in strings)
\font\eighttex=cmtex8
\fontdimen7\tentex=0pt % no double space after sentences
\fontdimen7\ninetex=0pt
\fontdimen7\eighttex=0pt

\newif\iftenpoint
\def\tenpoint{\tenpointtrue
 \def\rm{\fam0\tenrm}%
  \textfont0=\tenrm \scriptfont0=\sevenrm \scriptscriptfont0=\fiverm
  \textfont1=\teni \scriptfont1=\seveni \scriptscriptfont1=\fivei
  \textfont2=\tensy \scriptfont2=\sevensy \scriptscriptfont2=\fivesy
  \textfont3=\tenex \scriptfont3=\tenex \scriptscriptfont3=\tenex
  \def\it{\fam\itfam\tenit}%
  \textfont\itfam=\tenit
  \def\sl{\fam\slfam\tensl}%
  \textfont\slfam=\tensl
  \def\bf{\fam\bffam\tenbf}%
  \textfont\bffam=\tenbf \scriptfont\bffam=\sevenbf
   \scriptscriptfont\bffam=\fivebf
  \def\tt{\fam\ttfam\tentt}%
  \textfont\ttfam=\tentt
  \def\ttx{\tentex}%
  \normalbaselineskip=12pt
  \let\cmntfont=\tenrm
  \let\mc=\ninerm
  \let\sc=\eightrm
  \let\big=\tenbig
  \setbox\strutbox=\hbox{\vrule height8pt depth3pt width 0pt}%
  \normalbaselines\rm}

\def\ninepoint{\tenpointfalse
 \def\rm{\fam0\ninerm}%
  \textfont0=\ninerm \scriptfont0=\sixrm \scriptscriptfont0=\fiverm
  \textfont1=\ninei \scriptfont1=\sixi \scriptscriptfont1=\fivei
  \textfont2=\ninesy \scriptfont2=\sixsy \scriptscriptfont2=\fivesy
  \textfont3=\tenex \scriptfont3=\tenex \scriptscriptfont3=\tenex
  \def\it{\fam\itfam\nineit}%
  \textfont\itfam=\nineit
  \def\sl{\fam\slfam\ninesl}%
  \textfont\slfam=\ninesl
  \def\bf{\fam\bffam\ninebf}%
  \textfont\bffam=\ninebf \scriptfont\bffam=\sixbf
   \scriptscriptfont\bffam=\fivebf
  \def\tt{\fam\ttfam\ninett}%
  \textfont\ttfam=\ninett
  \def\ttx{\ninetex}%
  \normalbaselineskip=11pt
  \let\cmntfont=\ninerm
  \let\mc=\eightrm
  \let\sc=\sevenrm
  \let\big=\ninebig
  \setbox\strutbox=\hbox{\vrule height8pt depth3pt width 0pt}%
  \normalbaselines\rm}

\def\eightpoint{%
 \def\rm{\fam0\eightrm}%
  \textfont0=\eightrm \scriptfont0=\sixrm \scriptscriptfont0=\fiverm
  \textfont1=\eighti \scriptfont1=\sixi \scriptscriptfont1=\fivei
  \textfont2=\eightsy \scriptfont2=\sixsy \scriptscriptfont2=\fivesy
  \textfont3=\tenex \scriptfont3=\tenex \scriptscriptfont3=\tenex
  \def\it{\fam\itfam\eightit}%
  \textfont\itfam=\eightit
  \def\sl{\fam\slfam\eightsl}%
  \textfont\slfam=\eightsl
  \def\bf{\fam\bffam\eightbf}%
  \textfont\bffam=\eightbf \scriptfont\bffam=\sixbf
   \scriptscriptfont\bffam=\fivebf
  \def\tt{\fam\ttfam\eighttt}%
  \textfont\ttfam=\eighttt
  \def\ttx{\eighttex}%
  \normalbaselineskip=9pt
  \let\cmntfont=\eightrm
  \let\mc=\sevenrm
  \let\sc=\sixrm
  \let\big=\eightbig
  \setbox\strutbox=\hbox{\vrule height7pt depth2pt width 0pt}%
  \normalbaselines\rm}

\tenpoint
\def\tenbig#1{{\hbox{$\left#1\vbox to8.5pt{}\right.\nulldelimiterspace=0pt$}}}
\def\ninebig#1{{\hbox{$\textfont0=\tenrm\textfont2=\tensy
  \left#1\vbox to7.25pt{}\right.\nulldelimiterspace=0pt$}}}
\def\eightbig#1{{\hbox{$\textfont0=\ninerm\textfont2=\ninesy
  \left#1\vbox to6.5pt{}\right.\nulldelimiterspace=0pt$}}}

%\font\tenss=cmss10 \let\cmntfont\tenss % alternative comment font
\font\titlefont=cmr7 scaled\magstep4 % title on the contents page
\font\ttitlefont=cmtt10 scaled\magstep2 % typewriter type in title

\def\\#1{\leavevmode\hbox{\it#1\/\kern.05em}} % italic type for identifiers
\def\|#1{\leavevmode\hbox{$#1$}} % one-letter identifiers look better this way
\def\&#1{\leavevmode\hbox{\bf
  \def\_{\kern.04em\vbox{\hrule width.3em height .6pt}\kern.08em}%
  #1\/\kern.05em}} % boldface type for reserved words
\def\.#1{\leavevmode\hbox{\ttx % typewriter type for strings
  \let\\=\BS % backslash in a string
  \let\{=\LB % left brace in a string
  \let\}=\RB % right brace in a string
  \let\~=\TL % tilde in a string
  \let\ =\SP % space in a string
  \let\_=\UL % underline in a string
  \let\&=\AM % ampersand in a string
  \let\^=\CF % circumflex in a string
  #1\kern.05em}}
\def\){{\tentex\kern-.05em}\discretionary{\hbox{\tentex\BS}}{}{}}
\def\AT{@} % at sign for control text (not needed in versions >= 2.9)
\def\ATL{\par\noindent\bgroup\catcode`\_=12 \postATL} % print @l in limbo
\def\postATL#1 #2 {\bf letter \\{\uppercase{\char"#1}}
   tangles as \tentex "#2"\egroup\par}
\def\noATL#1 #2 {}
\def\noatl{\let\ATL=\noATL} % suppress output from @l
\def\ATH{{\pdffalse\X\kern-.5em:Preprocessor definitions\X}}
\let\PB=\relax % hook for program brackets |...| in TeX part or section name

\chardef\AM=`\& % ampersand character in a string
\chardef\BS=`\\ % backslash in a string
\chardef\LB=`\{ % left brace in a string
\chardef\RB=`\} % right brace in a string
\def\SP{{\tt\char`\ }} % (visible) space in a string
\chardef\TL=`\~ % tilde in a string
\chardef\UL=`\_ % underline character in a string
\chardef\CF=`\^ % circumflex character in a string

\newbox\PPbox % symbol for ++
\setbox\PPbox=\hbox{\kern.5pt\raise1pt\hbox{\sevenrm+\kern-1pt+}\kern.5pt}
\def\PP{\copy\PPbox}
\newbox\MMbox \setbox\MMbox=\hbox{\kern.5pt\raise1pt\hbox{\sevensy\char0
 \kern-1pt\char0}\kern.5pt}
\def\MM{\copy\MMbox}
\newbox\MGbox % symbol for ->
\setbox\MGbox=\hbox{\kern-2pt\lower3pt\hbox{\teni\char'176}\kern1pt}
\def\MG{\copy\MGbox}
\def\MRL#1{\mathrel{\let\K==#1}}
\let\GG=\gg
\let\LL=\ll
\let\NULL=\Lambda
\mathchardef\AND="2026 % bitwise and; also \& (unary operator)
\let\OR=\mid % bitwise or
\let\XOR=\oplus % bitwise exclusive or
\def\CM{{\sim}} % bitwise complement
\newbox\MODbox \setbox\MODbox=\hbox{\eightrm\%}
\def\MOD{\mathbin{\copy\MODbox}}
\def\DC{\kern.1em{::}\kern.1em} % symbol for ::
\def\PA{\mathbin{.*}} % symbol for .*
\def\MGA{\mathbin{\MG*}} % symbol for ->*
\def\this{\&{this}}

\newbox\bak \setbox\bak=\hbox to -1\em{} % backspace one em
\newbox\bakk\setbox\bakk=\hbox to -2\em{} % backspace two ems

\newcount\ind % current indentation in ems
\def\1{\global\advance\ind by1\hangindent\ind\em} % indent one more notch
\def\2{\global\advance\ind by-1} % indent one less notch
\def\3#1{\hfil\penalty#10\hfilneg} % optional break within a statement
\def\4{\copy\bak} % backspace one notch
\def\5{\hfil\penalty-1\hfilneg\kern2.5\em\copy\bakk\ignorespaces}% optional break
\def\6{\ifmmode\else\par % forced break
  \hangindent\ind\em\noindent\kern\ind\em\copy\bakk\ignorespaces\fi}
\def\7{\Y\6} % forced break and a little extra space
\def\8{\hskip-\ind\em\hskip 2\em} % no indentation

\newcount\gdepth % depth of current major group, plus one
\newtoks\gtitle % title of current major group
\newskip\intersecskip \intersecskip=12pt minus 3pt % space between sections
\let\yskip=\smallskip
\def\?{\mathrel?}
\def\,{\relax\ifmmode\mskip\thinmuskip\else\thinspace\fi}
\def\note#1#2.{\par\penalty5000
  \Y\noindent{\hangindent2\em\baselineskip10pt%
    \eightrm#1~\ifacro{\pdfnote#2.}\else#2\fi.\par}}

\newtoks\toksA \newtoks\toksB \newtoks\toksC \newtoks\toksD
\newtoks\toksE \newtoks\toksF \newtoks\usersanitizer
\newcount\countA \countA=0 \newcount\countB \countB=0
\newcount\countC \countC=0 \newcount\countD \countD=0
\newif\iftokprocessed \newif\ifTnum \newif\ifinstr
{\def\\{\global\let\spacechar= }\\ }

\ifacro % The following are pdf macros
\def\thewidth{\the\wd0 \space}
\def\theheight{\the\ht\strutbox\space}
\def\thedepth{\the\dp\strutbox\space}
\ifpdftex
  \ifx\pdfannotlink\undefined\let\pdfannotlink\pdfstartlink\fi% for pdfTeX 0.14
  \def\pdflink#1#2{\hbox{\pdfannotlink height\ht\strutbox depth\dp\strutbox
    attr{/Border [0 0 0]} goto num #1 \PDFlinkcolor #1\Black\pdfendlink}}
\else\def\pdflink#1#2{\setbox0=\hbox{\special{pdf: bc [ \pdflinkcolor ]}{#1}%
    \special{pdf: ec}}\special{pdf: ann width \thewidth height \theheight
      depth \thedepth << /Type /Annot /Subtype /Link
      /Border [0 0 0] /A << /S /GoTo /D (#2) >> >>}\box0\relax}\fi
\def\pdfnote#1.{\setbox0=\hbox{\toksA={#1.}\toksB={}\maketoks}\the\toksA}
\def\firstsecno#1.{\setbox0=\hbox{\toksA={#1.}\toksB={}%
    \def\makenote{\addtokens\toksB{\the\toksC}\def\makenote{\toksD={}
      \toksC={}\let\space\empty}\makenote}\maketoks}}
\def\addtokens#1#2{\edef\addtoks{\noexpand#1={\the#1#2}}\addtoks}
\def\poptoks#1#2|ENDTOKS|{\let\first=#1\toksD={#1}%
  \ifcat\noexpand\first0\countB=`#1\else\countB=0\fi\toksA={#2}}
\def\maketoks{\expandafter\poptoks\the\toksA|ENDTOKS|%
  \ifnum\countB>`9 \countB=0 \fi
  \ifnum\countB<`0
    \ifnum0=\countC\else\makenote\fi
    \ifx\first.\let\next=\maketoksdone\else
        \let\next=\maketoks
        \addtokens\toksB{\the\toksD}
        \ifx\first,\addtokens\toksB{\space}\fi
    \fi
  \else \addtokens\toksC{\the\toksD}\global\countC=1\let\next=\maketoks
  \fi
  \next
}
\def\makenote{\addtokens\toksB
    {\noexpand\pdflink{\the\toksC}{\romannumeral\the\toksC}}\toksC={}\global\countC=0}
\def\maketoksdone{\edef\st{\global\noexpand\toksA={\the\toksB}}\st}
\def\pdfURL#1#2{\ifpdf \ifpdftex\pdfannotlink height \theheight depth \thedepth
    attr {/Border [0 0 0]} user { /Type /Annot /Subtype /Link /A
      << /S /URI /URI (#2) >>}\Blue #1\Black \pdfendlink % changed in 3.69
  \else {\setbox0=\hbox{\special{pdf: bc [ \pdflinkcolor ]}{#1}%
    \special{pdf: ec}}\special{pdf: ann width \thewidth height \theheight
      depth \thedepth << /Border [0 0 0] /Type /Annot /Subtype /Link /A
      << /S /URI /URI (#2) >> >>}\box0\relax}\fi
  \else #1 ({\tt#2})\fi}
{\catcode`\~=12 \gdef\TILDE/{~}} % ~ in a URL
{\catcode`\_=12 \gdef\UNDER/{_}} % _ in a URL
\def\sanitizecommand#1#2{\addtokens\usersanitizer
       {\noexpand\dosanitizecommand\noexpand#1{#2}}}
\def\dosanitizecommand#1#2{\ifx\nxt#1\addF{#2}\fi}

\catcode`\[=1 \catcode`\]=2 \catcode`\{=12 \catcode`\}=12
  \def\lbchar[{]  \def\rbchar[}]
\catcode`\[=12 \catcode`\]=12 \catcode`\{=1 \catcode`\}=2
\catcode`\~=12 \def\tildechar{~}  \catcode`\~=13
\catcode`\|=0 |catcode`|\=12 |def|bschar{\} |catcode`|\=0 \catcode`\|=12
\def\makeoutlinetoks{\Tnumfalse\afterassignment\makeolproctok\let\nxt= }
\def\makeolnexttok{\afterassignment\makeolproctok\let\nxt= }
\def\makeolgobbletok{\afterassignment\makeolnexttok\let\nxt= }
\def\addF#1{\addtokens\toksF{#1}\tokprocessedtrue}
% now comes a routine to "sanitize" section names, for pdf outlines
\def\makeolproctok{\tokprocessedfalse
  \let\next\makeolnexttok % default
  \ifx\nxt\outlinedone\let\next\outlinedone
  \else\ifx{\nxt \else\ifx}\nxt \Tnumfalse \instrfalse % skip braces
  \else\ifx$\nxt % or a $ sign
  \else\ifx^\nxt \addF^\else\ifx_\nxt \addF_% sanitize ^ and _
  \else\ifx\nxt\spacechar \addF\space
  \else\if\noexpand\nxt\relax % we have a control sequence; is it one we know?
    \ifx\nxt~\addF\space
    \else\ifx\nxt\onespace\addF\space
    \else\the\usersanitizer
    \iftokprocessed\else\makeolproctokctli
    \iftokprocessed\else\makeolproctokctlii
    \iftokprocessed\else\makeolproctokctliii % if not recognised, skip it
    \fi\fi\fi\fi\fi
   \else  % we don't have a control sequence, it's an ordinary char
    \ifx/\nxt \addF{\string\/}% quote chars special to PDF with backslash
    \else\ifx(\nxt \addF{\string\(}\else\ifx)\nxt \addF{\string\)}%
    \else\ifx[\nxt \addF{\string\[}\else\ifx]\nxt \addF{\string\]}%
    \else\expandafter\makeolproctokchar\meaning\nxt
   \fi\fi\fi\fi\fi\fi\fi\fi\fi\fi\fi\fi\fi
  \next
}
\def\makeolproctokchar#1 #2 #3{\addF{#3}}
\def\makeolproctokctli{%
  \ifx\nxt\CEE\addF{C}\let\next\makeolgobbletok % \CEE/
  \else\ifx\nxt\UNIX\addF{UNIX}\let\next\makeolgobbletok % \UNIX/
  \else\ifx\nxt\TEX\addF{TeX}\let\next\makeolgobbletok % \TEX/
  \else\ifx\nxt\TeX\addF{TeX}\else\ifx\nxt\LaTeX\addF{LaTeX}%
  \else\ifx\nxt\CPLUSPLUS\addF{C++}\let\next\makeolgobbletok % \CPLUSPLUS/
  \else\ifx\nxt\Cee\addF{C}%
  \else\ifx\nxt\PB \let\next\makeolgobbletok \tokprocessedtrue % \PB{...}
  \else\ifx\nxt\.\tokprocessedtrue\instrtrue % \.{...}
      % skip \|
  \else\ifx\nxt\\\ifinstr\addF{\bschar\bschar}\else\tokprocessedtrue\fi
  \else\ifx\nxt\&\ifinstr\addF&\else\tokprocessedtrue\fi
  \else\ifx\nxt\~\ifTnum\addF{0}\else\addF\tildechar\fi % 077->\T{\~77}
  \else\ifx\nxt\_\ifTnum\addF{E}\else\addF_\fi  % 0.1E5->\T{0.1\_5}
  \else\ifx\nxt\^\ifTnum\addF{0x}\else\addF^\fi  % 0x77 -> \T{\^77}
  \else\ifx\nxt\\\\\ifTnum\addF{0b}\else\addF\\\fi  % 0b10111 -> \T{\\10111}
  \else\ifx\nxt\$\ifTnum\tokprocessedtrue\else\addF$\fi %$% \T{77\$L}
  \else\ifx\nxt\{\addF\lbchar       \else\ifx\nxt\}\addF\rbchar
  \else\ifx\nxt\ \addF\space        \else\ifx\nxt\#\addF{\string\#}%
  \else\ifx\nxt\PP\addF{++}\else\ifx\nxt\MM\addF{--}%
  \fi\fi\fi\fi\fi\fi\fi\fi\fi\fi\fi\fi\fi\fi\fi\fi\fi\fi\fi\fi\fi\fi
}
\def\makeolproctokctlii{%
  \ifx\nxt\MG\addF{->}\else\ifx\nxt\GG\addF{>>}%
  \else\ifx\nxt\LL\addF{<<}\else\ifx\nxt\NULL\addF{NULL}%
  \else\ifx\nxt\AND\addF&\else\ifx\nxt\OR\addF|%
  \else\ifx\nxt\XOR\addF^\else\ifx\nxt\CM\addF\tildechar
  \else\ifx\nxt\MOD\addF{\string\%}\else\ifx\nxt\DC\addF{::}%
  \else\ifx\nxt\PA\addF{.*}\else\ifx\nxt\MGA\addF{->*}%
  \else\ifx\nxt\this\addF{this}\else\ifx\nxt\?\addF?%
  \else\ifx\nxt\E\addF{==}\else\ifx\nxt\G\addF{>=}%
  \else\ifx\nxt\I\addF{!=}\else\ifx\nxt\K\addF{=}%
  \else\ifx\nxt\l\addF{l}\else\ifx\nxt\L\addF{L}%
  \else\ifx\nxt\o\addF{o}\else\ifx\nxt\O\addF{O}%
  \else\ifx\nxt\R\addF!%
  \else\ifx\nxt\T \Tnumtrue \let\next\makeolgobbletok
    \tokprocessedtrue % \T{number}
  \else\ifx\nxt\AM\addF&\else\ifx\nxt\%\addF{\string\%}%
  \fi\fi\fi\fi\fi \fi\fi\fi\fi\fi
  \fi\fi\fi\fi\fi \fi\fi\fi\fi\fi
  \fi\fi\fi\fi\fi\fi
}
\def\makeolproctokctliii{%
  \ifx\nxt\V\addF{||}\else\ifx\nxt\W\addF{&&}\else\ifx\nxt\Z\addF{<=}%
  \else\ifx\nxt\*\addF*\else\ifx\nxt\Xand\addF{\space and\space}%
  \else\ifx\nxt\Xandxeq\addF{\space and_eq\space}%
  \else\ifx\nxt\Xbitand\addF{\space bitand\space}%
  \else\ifx\nxt\Xbitor\addF{\space bitor\space}%
  \else\ifx\nxt\Xcompl\addF{\space compl\space}%
  \else\ifx\nxt\Xnot\addF{\space not\space}%
  \else\ifx\nxt\Xnotxeq\addF{\space not_eq\space}%
  \else\ifx\nxt\Xor\addF{\space or\space}%
  \else\ifx\nxt\Xorxeq\addF{\space or_eq\space}%
  \else\ifx\nxt\Xxor\addF{\space xor\space}%
  \else\ifx\nxt\Xxorxeq\addF{\space xor_eq\space}%
  \fi\fi\fi\fi\fi\fi\fi\fi\fi\fi
  \fi\fi\fi\fi\fi
}
\def\outlinedone{\edef\outlinest{\global\noexpand\toksE={\the\toksF}}%
  \outlinest\let\outlinedone=\relax}
\fi % End of pdf macros

\def\lapstar{\rlap{*}}
\def\stsec{\tenpoint\rightskip=0pt % get out of C mode (cf. \B)
  \sfcode`;=1500 \pretolerance 200 \hyphenpenalty 50 \exhyphenpenalty 50
  \noindent\strut{\let\*=\lapstar\bf\secno.\quad}%
  \ifacro \smash{\raise\baselineskip\hbox to0pt{\let\*=\empty
    \ifpdftex \pdfdest num \secstar fith%
    \else \special{pdf: dest (\romannumeral\secstar)
      [ @thispage /FitH @ypos ]}\fi}}\fi}
\def\startsection{\titletrue
  \line{\smash{\titlefont\title}\quad\hrulefill}\bigskip
  \let\startsection=\stsec\stsec}
\def\defin#1{\global\advance\ind by 2 \1\&{#1 }} % begin `define' or `format'
\def\A{\note{See also section}} % xref for doubly defined section name
\def\As{\note{See also sections}} % xref for multiply defined section name
\def\B{\iftenpoint\ninepoint\fi
  \font\niness=cmss9 \let\cmntfont\niness % alternative comment font
  \rightskip=0pt plus 100pt minus 10pt % go into C mode
  \sfcode`;=3000
  \pretolerance 10000
  \hyphenpenalty 9999 % so strings can be broken (discretionary \ is inserted)
  \exhyphenpenalty 10000
  \global\ind=2 \1\ \unskip}
%\def\C#1{\5\5\quad$/\ast\,${\cmntfont #1}$\,\ast/$}
%\let\SHC\C % "// short comments" treated like "/* ordinary comments */"
\def\C#1{\5\5\quad$\triangleright\,${\cmntfont#1}$\,\triangleleft$}
\def\SHC#1{\5\5\quad$\diamond\,${\cmntfont#1}}
\def\D{\defin{{\rm\#}define}} % macro definition
\let\E=\equiv % equivalence sign
\def\ET{ and~} % conjunction between two section numbers
\def\ETs{, and~} % conjunction between the last two of several section numbers
\def\F{\defin{format}} % format definition
\let\G=\ge % greater than or equal sign
% \H is long Hungarian umlaut accent
\let\I=\ne % unequal sign
\def\J{\.{@\&}} % TANGLE's join operation
%\let\K== % assignment operator
\let\K=\leftarrow % "honest" alternative to standard assignment operator
% \L is Polish letter suppressed-L
\outer\def\M#1{%
  {\xdef\secstar{#1}\let\*=\empty\xdef\secno{#1}}% remove \* from section name
  \vskip\intersecskip\startsection\ignorespaces}
\outer\def\N{\ifvoid\partialpage\lefttrue
  \else\ifdim\ht\partialpage<\pageht \leftfalse\else\lefttrue\fi\fi\NNN}
\outer\def\NN{\ifvoid\partialpage\leftfalse
  \else\ifdim\ht\partialpage<\pageht \lefttrue\else\leftfalse\fi\fi\NNN}
\outer\def\NNN#1#2#3.{% beginning of starred section
  \ifacro{\toksF={}\makeoutlinetoks#3\outlinedone\outlinedone}\fi
  \gdepth=#1\gtitle={#3}\def\secno{#2}\xdef\secstar{#2}
  \ifleft \flushout
    \gdef\rlhead{\let\i=I\uppercase{\ignorespaces#3}} % running left headline
    \global\let\rrhead=\rlhead % running right headline
  \else\global\setbox\partialpage=\vbox{
          \vbox to\pageht{\unvbox\partialpage\vfill}\break}
      \gdef\rrhead{\let\i=I\uppercase{\ignorespaces#3}}
  \fi
  \message{*\secno} % progress report
  \ifpdf \ifpdftex\def\stripprefix##1>{}\def\gtitletoks{#3}%
    \edef\gtitletoks{\expandafter\stripprefix\meaning\gtitletoks}%
    {\immediate\write\cont{\ZZ{\gtitletoks}{#1}{\secno}% write to contents file
      {\noexpand\the\pageno}{\the\toksE}}} % \ZZ{title}{depth}{sec}{page}{ss}
    \expandafter\xdef\csname curr#1\endcsname{\secno}
    \ifnum#1>0\countB=#1 \advance\countB by-1
      \advancenumber{chunk\the\countB.\expnumber{curr\the\countB}}\fi
  \else \special{pdf: outline #1 << /Title (\the\toksE) /Dest
    [ @thispage /FitH @ypos ] >>}\fi\fi
  \startsection{\bf\ignorespaces#3.\quad}\ignorespaces}
\def\expnumber#1{\expandafter\ifx\csname#1\endcsname\relax 0%
  \else \csname#1\endcsname \fi} % Petr Olsak's macros from texinfo.tex
\def\advancenumber#1{\countA=\expnumber{#1}\relax \advance\countA by1
  \expandafter\xdef\csname#1\endcsname{\the\countA}}
% \O is Scandinavian letter O-with-slash
% \P is paragraph sign
\def\Q{\note{This code is cited in section}} % xref for mention of a section
\def\Qs{\note{This code is cited in sections}} % xref for mentions of a section
\let\R=\lnot % logical not
% \S is section sign
\def\T#1{\leavevmode % octal, hex or decimal constant
  \hbox{$\def\?{\kern.2em}%$%
    \let\ \, % C++ digit separator becomes a little white space
    \def\$##1{\egroup_{\rm##1}\bgroup}% suffix to constant
    \def\_{\cdot 10^{\aftergroup}}% power of ten (via dirty trick)
    \let\~=\oct \let\^=\hex \let\\=\bin {#1}$}}%$%
\def\U{\note{This code is used in section}} % xref for use of a section
\def\Us{\note{This code is used in sections}} % xref for uses of a section
\let\V=\lor % logical or
\let\W=\land % logical and
\def\X#1:#2\X{\ifmmode\gdef\XX{\null$\null}\else\gdef\XX{}\fi %$% section name
  \XX$\langle\,${\let\I=\ne#2\sevenrm\kern.5em
    \ifacro{\pdfnote#1.}\else#1\fi}$\,\rangle$\XX}
\def\Y{\par\yskip}
\let\Z=\le
\let\ZZ=\let % now you can \write the control sequence \ZZ
\let\*=*

%\def\oct{\hbox{\rm\char'23\kern-.2em\it\aftergroup\?\aftergroup}} % WEB style
%\def\hex{\hbox{\rm\char"7D\tt\aftergroup}} % WEB style
\def\oct{\hbox{$^\circ$\kern-.1em\it\aftergroup\?\aftergroup}} % CWEB style
\def\hex{\hbox{$^{\scriptscriptstyle\#}$\tt\aftergroup}} % CWEB style
\def\bin{\hbox{$^{\scriptscriptstyle b}$\tt\aftergroup}} % new in CWEB 4.3
\def\vb#1{\leavevmode\hbox{\kern2pt\vrule\vtop{\vbox{\hrule
        \hbox{\strut\kern2pt\.{#1}\kern2pt}}
      \hrule}\vrule\kern2pt}} % verbatim string
\def\p#1{\cdot 2^{#1}} % power of two (hex exponent)

% now here's the mini-index formatting control
\newcount\nrefs % total number of references in partial page
\newcount\baseno % smallest section number in partial page
\toksdef\prefs=199 % references to previous sections in current program
\toksdef\frefs=220 % references to future sections in current program
\toksdef\erefs=221 % references to sections in another program
% \count and \toks registers 200--219 are also used to keep track of refs
\catcode`\@=11
\newcount\@n \newcount\@m \newcount\@p
\newdimen\pageht \pageht=19cm
\newdimen\pagewd \pagewd=13cm
\newdimen\colwd \colwd=\pagewd
 \advance\colwd by -2pc \divide\colwd by 3 % for three columns
\newdimen\fullpageht \fullpageht=\pageht \advance\fullpageht by 4pc
\newdimen\pagethresh \pagethresh=2\pageht
\newdimen\nsize \newdimen\msize
\newskip\intersecskip \intersecskip=8pt plus 2pt minus 3pt
\newbox\partialpage
\newbox\newsec
\newtoks\newrefs
\newtoks\ttoks
\newdimen\ninept \ninept=9pt
\newif\iftitle \newif\ifleft

\def\flushout{\ifvoid\partialpage\else
    \setbox0=\vsplit\partialpage to \pageht
    \shipout\vbox{
      \vbox to 3pc{\leftheadline\vfill}
      \nointerlineskip\box0}
    \global\advance\pageno 1
    \global\let\[=\makeinref \global\let\]=\makeoutref
    \ifsorted \let\readin=\readln \readrefs
    \else \immediate\write\reffile{!\the\pageno}\let\readin=\readrefs \fi
    \setbox0=\vbox{\eightpoint \hsize=\colwd
      \rightskip=0pt plus 100pt minus 10pt
      \pretolerance 10000
      \hyphenpenalty 10000 \exhyphenpenalty 10000
      \noindent\vbox to1pt{}\par % 1pt = \topskip - \ninept
      \readin}
    \shipout\vbox to\fullpageht{
      \vbox to 3pc{\rightheadline\vfill}
      \unvbox\partialpage
      \vfill
      \ifdim\ht0>1pt \kern11pt\hrule \hbox{%
          \nsize=\ht0 \advance\nsize-\topskip
          \divide\nsize by 3 \divide\nsize by\ninept
          \multiply\nsize by\ninept \advance\nsize\topskip
          \vsplit0 to\nsize \kern1pc
          \msize=\ht0 \advance\msize-\topskip
          \divide\msize by 2 \divide\msize by\ninept
          \multiply\msize by\ninept \advance\msize\topskip
          \vbox to\nsize{\vsplit0 to\msize\vss}\kern1pc
          \vbox to\nsize{\box0\vss}} \fi}
    \global\advance\pageno 1 \global\let\rlhead=\rrhead\fi
  {\globaldefs=1
    \@n=199 \loop \ifnum\@n<222 \toks\@n={} \count\@n=0 \advance\@n 1 \repeat
    \nrefs=0
    \baseno=\secno}}
\def\leftheadline{\hbox to\pagewd{\vbox to10pt{}%
  \iftitle\global\titlefalse\else\ninerm\title:\enspace \eightsl\rlhead\fi
  \hfil\eightrm\folio}}
\def\rightheadline{\hbox to\pagewd{\vbox to10pt{}%
  \eightrm\folio\hfil\ninerm\title:\enspace \eightsl\rrhead\/}}

\def\mini{\begingroup \obeylines \globaldefs=1 \newrefs=\bgroup}
\def\FI{\@n=\secno \advance\@n-\baseno \advance\@n 200
  \advance\nrefs-\count\@n \@n=\nrefs
  \let\[=\countnewref \let\]=\cn \the\newrefs \endgroup
  \nobreak\null\nobreak\vskip-\baselineskip\penalty-500\vfil\eject}
\def\twillout{\setbox\newsec=\vbox{\unvbox255}
  \ifnum\@n=0 \nsize=0pt
  \else \nsize=3pt \multiply\nsize\@n \advance\nsize 1.25pc \fi
  \ifvoid\partialpage\else\advance\nsize 5pt \advance\nsize\ht\partialpage\fi
  \advance\nsize\ht\newsec 
  \message{\the\nsize} % show the page break decision amount
  \ifbr \global\brfalse \flushout \message{/\secno}\fi
  \ifdim\nsize>\pagethresh \flushout \message{\secno}\fi
  \addtopartialpage}
\def\addtopartialpage{\globaldefs=1
  \let\[=\addnewref \let\]=\addneweref \the\newrefs
  \setbox\partialpage=\ifvoid\partialpage \box\newsec
  \else \vbox{\unvbox\partialpage \vfilneg
      \vskip\intersecskip \unvbox\newsec}\fi}
\def\shortpage{\par\vskip-2\baselineskip
  \null\vfill\penalty-5000\message{(shortpage)}}
\newif\ifbr
\def\forcebr{\global\brtrue}

\def\countnewref #1 {\ifnum#1<\baseno \let\next\cn
  \else\ifnum#1>\secno \let\next\cn \else\let\next\flushref\fi\fi\next#1 }
\def\cn #1 #2 {\begintest{#1 #2}\cnr\endtest}

% here's a tricky way to present \cnr with a csname unexpanded inside:
\def\begintest#1{\def\TEST{#1}\expandafter\expandafter\expandafter}
\def\endtest{\csname\expandafter\gobb\meaning\TEST\endcsname}
\expandafter\def\expandafter\gobb\meaning\empty{}

{\obeylines\gdef\cnr#1#2
  {\ifx#1\relax \advance\@n 1 \fi}}
{\obeylines\gdef\flushref#1
  {}}
\def\addnewref #1 {\ifnum#1<\baseno \let\next\an \@n=199
  \else\ifnum#1>\secno \let\next\an \@n=#1 \advance\@n-\baseno
    \advance\@n200 \ifnum\@n>219 \@n=220 \fi
    \else\let\next\flushref\fi\fi\next#1 }
\def\an#1 #2 {\begintest{#1 #2}\anr\endtest\[#1 #2 }
\def\addneweref#1 #2 {\@n=221 \begintest{#1 #2}\anr\endtest\]#1 #2 }
{\obeylines\gdef\anr#1#2
  {\ifx#1\relax \advance\nrefs1 \advance\count\@n 1 %
  \def#1{#2
    }\toks\@n=\expandafter{\the\toks\@n\lmda#1}\fi}}
\ifsorted \def\lmda#1{\global\let#1\relax}
\else \def\lmda#1{#1\global\let#1\relax} \fi
{\obeylines\gdef\makeinref#1 #2 #3
  {\ifsorted\else\ttoks={\[#1 #2 #3}\immediate\write\reffile{+ \the\ttoks}\fi %
  \hangindent=1em \noindent #2\miniform#3, %<== space!
    \S\ifacro{\pdfnote#1.}\else{#1}\fi.\par}}
{\obeylines\gdef\makeoutref#1 #2 #3
  {\ifsorted\else\ttoks={\]#1 #2 #3}\immediate\write\reffile{+ \the\ttoks}\fi %
  \hangindent=1em \noindent #2\miniform#3, \extref#1.\par}}
\def\miniform{\futurelet\next\miniforma}
\def\miniforma{\ifx\next\zip
  \else\if=\noexpand\next \let\next\gobbleq
    \else\let\next\colon\fi\fi\next}
\def\zip{}
\def\gobbleq={\,=\,}
\def\colon{: }
\def\extref{\futurelet\next\extrefa}
\def\extrefa{\if"\next\let\next\gobblest\else\let\next\cite\fi\next}
\def\gobblest"#1"{{\ttx#1}}
\def\cite#1#2.{{\sc#1}\,\S#2.}
\def\uninitialized{???}

\def\readrefs{\the\prefs \@m=\secno \advance\@m-\baseno \advance\@m200
  {\def\lmda##1{\global\let##1\relax}\@p=200
    \loop\ifnum\@p<\@m \the\toks\@p \advance\@p 1 \repeat}
  \loop\ifnum\@m<222 \the\toks\@m \advance\@m 1 \repeat}
\def\readln{\read\sreffile to\next \expandafter\next\newline \readin}
{\obeylines\gdef\newline{\unskip\noexpand
  }}
\def\donewithpage#1 {\let\readin=\relax}
\catcode`\@=12

\hsize=\pagewd \vsize=\maxdimen \output={\twillout}

%\let\page=\pagebody \raggedbottom
% \def\page{\box255 }\normalbottom % faster, but loses plain TeX footnotes
%\def\normaloutput#1#2#3{\ifodd\pageno\hoffset=\pageshift\fi
 %\shipout\vbox{
  %\vbox to\fullpageht{
  %\iftitle\global\titlefalse
  %\else\hbox to\pagewd{\vbox to10pt{}\ifodd\pageno #3\else#2\fi}\fi
  %\vfill#1}} % parameter #1 is the page itself
  %\global\advance\pageno by1}
%\def\topofcontents{\centerline{\titlefont\title}\vskip.7in
  %\vfill} % this material will start the table of contents page
%\def\startpdf{\ifpdf \ifpdftex\pdfcatalog{/PageMode /UseOutlines}
  %\else {\special{pdf: docview << /PageMode /UseOutlines >>}}\fi\fi}
%\def\botofcontents{\vfill
  %\centerline{\covernote}} % this material will end the table of contents page
%\def\covernote{}
\def\contentspagenumber{0} % default page number for table of contents
%\newdimen\pageshift \pageshift=\hoffset
   %% shift righthand pages wrt lefthand ones (changed in version 3.70)
%\def\setpage{\hsize\pagewd\vsize\pageht} % use after changing page size
\def\contentsfile{\jobname.toc} % file that gets table of contents info
\def\readcontents{\input \contentsfile}
\def\readsections{\input \jobname.scn}

\newwrite\cont
\ifpdftex
  \immediate\openout\cont=\contentsfile
  \immediate\write\cont{\catcode `\noexpand\@=11\relax} % \makeatletter
\fi

\def\inx{\flushout % No full index at the end
  \ifpdftex
    \immediate\write\cont{\catcode `\noexpand\@=12\relax} % \makeatother
    \immediate\closeout\cont % the contents information has been fully gathered
  \fi}
\def\fin{\ifacro \message{Section names:}
  \def\U##1.{} \def\Us##1.{} \let\I=\relax \let\*=*
  \def\outsecname{Names of the sections}
  \ifpdftex \makebookmarks \countsections
    \pdfdest name {NOS} fith
    \pdfoutline goto name {NOS} count -\the\countD {\outsecname}
    \def\X##1:##2\X{\firstsecno##1.%
      {\toksF={}\makeoutlinetoks##2\outlinedone\outlinedone}%
      \pdfoutline goto num \the\toksA \expandafter{\the\toksE}}
  \else \output={\shipout\box255}
    \special{pdf: outline -1 << /Title (\outsecname) >>}%
    \def\X##1:##2\X{\firstsecno##1.%
      {\toksF={}\makeoutlinetoks##2\outlinedone\outlinedone}%
      \special{pdf: outline 0 << /Title (\the\toksE)
        /A << /S /GoTo /D (\romannumeral\the\toksA) >> >>}}
  \fi \readsections \fi}
\def\makebookmarks{\let\ZZ=\writebookmarkline \readcontents\relax}
\def\countsections{\message{Number of named sections:}
  {\def\I{\global\advance\countD by 1}\def\X##1\X{\relax}
  \def\Q##1.{\relax}\def\Qs##1.{\relax}\def\U##1.{\relax}\def\Us##1.{\relax}
  \readsections\relax}\message{\the\countD}}
\def\writebookmarkline#1#2#3#4#5{{%
  \let\(=\let \let\)=\let \let\[=\let \let\]=\let \let\/=\let
  \pdfoutline goto num #3 count -\expnumber{chunk#2.#3} {#5}}}
\let\con=\end % no table of contents
%\def\con{\par\vfill\eject % finish the section names
% \ifodd\pageno\else\titletrue\null\vfill\eject\fi % for duplex printers
  %\rightskip 0pt \hyphenpenalty 50 \tolerance 200
  %\setpage \output={\normaloutput\page\lheader\rheader}
  %\titletrue % prepare to output the table of contents
  %\pageno=\contentspagenumber
  %\def\grouptitle{TABLE OF CONTENTS}
  %\message{Table of contents:}
  %\topofcontents \startpdf
  %\line{\hfil Section\hbox to3em{\hss Page}}
  %\let\ZZ=\contentsline
  %\readcontents\relax % read the contents info
  %\botofcontents \end} % print the contents page(s) and terminate
%\def\contentsline#1#2#3#4#5{\ifnum#2=0 \smallbreak\fi
    %\line{\consetup{#2}#1
      %\rm\leaders\hbox to .5em{.\hfil}\hfil
      %\ \ifacro\pdflink{#3}{\romannumeral#3}\else#3\fi\hbox to3em{\hss#4}}}
%\def\consetup#1{\ifcase#1 \bf % depth -1 (@**)
  %\or % depth 0 (@*)
  %\or \hskip2em % depth 1 (@*1)
  %\or \hskip4em \or \hskip6em \or \hskip8em \or \hskip10em % depth 2,3,4,5
  %\else \hskip12em \fi} % depth 6 or more
\def\today{\ifcase\month\or
  January\or February\or March\or April\or May\or June\or
  July\or August\or September\or October\or November\or December\fi
  \space\number\day, \number\year}
\newcount\twodigits
\def\hours{\twodigits=\time \divide\twodigits by 60 \printtwodigits
  \multiply\twodigits by-60 \advance\twodigits by\time :\printtwodigits}
\def\gobbleone1{}
\def\printtwodigits{\advance\twodigits100
  \expandafter\gobbleone\number\twodigits
  \advance\twodigits-100 }
\def\datethis{\def\startsection{\leftline{\sc\today\ at \hours}\bigskip
  \let\startsection=\stsec\stsec}}
  % say `\datethis' in limbo, to get your listing timestamped before section 1
\def\datecontentspage{}

% To produce only a subset of pages, put the page numbers on separate
% lines in a file called pages.tex
\let\Shipout=\shipout
\newread\pages \newcount\nextpage \openin\pages=pages
\def\getnextpage{\ifeof\pages\else
 {\endlinechar=-1\read\pages to\next
  \ifx\next\empty % in this case we should have eof now
  \else\global\nextpage=\next\fi}\fi}
\ifeof\pages\else\message{OK, I'll ship only the requested pages!}
 \getnextpage\fi
\def\shipout{\ifeof\pages\let\next=\Shipout
 \else\ifnum\pageno=\nextpage\getnextpage\let\next=\Shipout
  \else\let\next=\Tosspage\fi\fi \next}
\newbox\garbage \def\Tosspage{\deadcycles=0\setbox\garbage=}
