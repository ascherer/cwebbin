% pdfwebtofront.tex
% Code from knuth.drv in https://github.com/oberdiek/latex-tds
% 2020/06/20 v2.0 by Andreas Scherer.
%
% This file is part of project https://github.com/ascherer/cwebbin
% and may be distributed under the MIT License or the LaTeX Project
% Public License.
%
% Move table-of-contents page to the front in PDF output.
% Works with pdftex and xetex in connection with either 'webmac.tex'
% for Pascal/WEB programs or 'cwebmac.tex' for C/CWEB programs.
%
\ifx\detokenize\undefined\endinput\fi
\def\contentsfile{\jobname.toc} % file that gets table of contents info
\newif\iftexmf\texmffalse % special toc treatment for TeX, Metafont, etc.
\newread\testread
\openin\testread=\contentsfile\relax
\ifeof\testread % First run
\else % Second run
  % Fix page numbers in the PDF bootmarks with 'Page Labels'
  \def\tocpages{1} % Most programs have one page Table-of-Contents
  \def\pagemode{/PageMode /UseOutlines}
  \def\pagelabels{/PageLabels << /Nums [
    \ifnum\contentspagenumber=0 0 << /P(Contents) >> \tocpages << /S/D/St 1 >>
    \else 0 << /S/D/St \contentspagenumber >> \fi ] >> }
  \def\startpdf{\ifpdf \ifpdflua\pdfcatalog{\pagemode \pagelabels}
    \else {\special{pdf: docview << \pagemode \pagelabels >>}}\fi\fi}
  % Redefine '\con' to be invoked before the first '\N' (starred section).
  \let\ORGcon\con
  \def\con{%
    % reduce size of PDF pages for more screen space
    \pdfpagewidth=\pagewidth \advance\pdfpagewidth by 1in
    \pdfpageheight=\fullpageheight \advance\pdfpageheight by 1in
    \ifpdflua \pdfhorigin=0.5in \pdfvorigin=0.5in
    \else \global\pageshift=-0.5in
      \global\hoffset=-0.5in \global\voffset=-0.5in \fi
    \begingroup
      \let\end\relax
      \ORGcon
    \endgroup
    \let\con\end
  }%
  % Special variant for 'mf.web' and 'tex.web' (et al.).
  % They all define a peculiar version of '\N' in their preamble.
  % And they put their tables-of-contents on a sparse 'page 2' and
  % start the main body on page '3'.
  \edef\x{\jobname}%
  \edef\y{\detokenize{tex}}%
  \ifx\x\y \texmftrue\else
  \edef\y{\detokenize{mf}}%
  \ifx\x\y \texmftrue\else
  \edef\y{\detokenize{pdftex}}%
  \ifx\x\y \texmftrue \def\tocpages{2}\else
  \edef\y{\detokenize{xetex}}%
  \ifx\x\y \texmftrue\fi\fi\fi\fi
  \iftexmf
    % These main programs start on page '3' (after any number of ToC pages)
    \def\pagelabels{/PageLabels << /Nums [
      0 << /P(Contents) >> \tocpages << /S/D/St 3 >> ] >> }
    \expandafter\let\csname ORGN\expandafter\endcsname
                    \csname N\endcsname
    \expandafter\outer\expandafter\def\csname N\endcsname{%
      \titletrue \con \pageno=2
      \expandafter\let\csname N\expandafter\endcsname
                      \csname ORGN\endcsname
      \csname N\endcsname
    }%
  \else % Much more general approach for WEB and CWEB programs.
    \edef\y{\detokenize{mp}}%
    % This main program starts on page '3' (after any number of ToC pages)
    \ifx\x\y \def\pagelabels{/PageLabels << /Nums [
        0 << /P(Contents) >> \tocpages << /S/D/St 3 >> ] >> } \fi
    % In fact, '\con' will be invoked _inside_ the first '\N' right _after_
    % '\MN', but _before_ '\eject'.  And '\topofcontents' gets in the twist
    % as well.
    \let\ORGtopofcontents\topofcontents \let\ORGrheader\rheader
    \def\topofcontents{%
      \def\:{\par\hangindent 2em} % Fix for 'bibtex.web'; from '\def\fin'.
      \ORGtopofcontents
      \let\topofcontents\relax
    }%
    % Squeeze '\con' between '\MN' and '\eject' in the very first '\N'.
    \expandafter\let\csname ORGvfil\expandafter\endcsname
                    \csname vfil\endcsname
    \expandafter\let\csname ORGeject\expandafter\endcsname
                    \csname eject\endcsname
    \expandafter\outer\expandafter\def\csname vfil\endcsname{%
      \let\eject\relax
      \titletrue % prepare to output the table of contents
      \topofcontents \con
      \advance\pageno by -1\relax
      \expandafter\let\csname vfil\expandafter\endcsname
                      \csname ORGvfil\endcsname
      \expandafter\let\csname eject\expandafter\endcsname
                      \csname ORGeject\endcsname
      \vfil\eject
      \let\rheader\ORGrheader
    }%
  \fi
\fi
